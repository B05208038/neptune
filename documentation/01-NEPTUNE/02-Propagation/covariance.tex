%------------------------------------------------------------------------------
\section{Covariance matrix propagation}
\label{sec:propagation-covariance}
%------------------------------------------------------------------------------

\subsection{State space representation}
\label{sec:state-space-notation}
%------------------------------------------------------------------------------

The orbit determination problem is typically approached by using a state space representation. For a very detailed description of the associated concepts, referencing the textbook
of \citet{tapley2004} is highly recommended, which also served as the basis for the following derivation. 

The \gls{sym:stateDim}-dimensional state vector $\xt$ of a satellite can be defined as
\begin{equation}
 \xt = \left(\rt^T, \vt^T, \gls{sym:cvec}^T \right)^T,
\end{equation}
with \gls{sym:cvec} being a multi-dimensional vector of constant parameters to be solved for in the orbit determination process, which might, for example,
include the drag and \acrshort{acr:srp} coefficients. Due to the non-linear nature of orbital motion, the conversion of the second-order differential equations of motion into a vector of first-order differential equations is written as
\begin{equation}
 \frac{d}{dt} \xt = \gls{sym:functionvec}\left(\gls{sym:state},\gls{sym:t}\right). \label{eq:non-linear-system}
\end{equation}
As the state, in general, cannot be observed directly, an additional equation, the \textit{measurement model} \citep{maybeck1979}, is required to map the state vector to the observations:
\begin{equation}
 \yt = \gls{sym:functiongvec}\left(\gls{sym:state},\gls{sym:t}\right) + \boldsymbol{\gls{sym:obsError}}, \label{eq:non-linear-obs}
\end{equation}
here, \gls{sym:obsError} is the error in the observation. With {\yt} being \gls{sym:obsDim}-dimensional, and, in general, $\gls{sym:obsDim}<\gls{sym:stateDim}$, for the orbit determination the number of 
observations \gls{sym:numObs} guarantees that $\gls{sym:numObs}\times\gls{sym:obsDim} >> \gls{sym:stateDim}$.

A linearization is obtained by introducing a reference trajectory, denoted by an index \gls{idx:ref}, and work with the deviations
\begin{align}
 \Delta\xt &= \xt - \xref, \label{eq:deviation-ref} \\
 \Delta\yt &= \yt - \yref, \label{eq:deviation-ref-y}
\end{align}
which are assumed to be small - a concept familiar to Encke's method for perturbed orbits. Expanding \eq{eq:non-linear-system} and \eq{eq:non-linear-obs} about the reference state gives
\begin{align}
 \xdot &= \gls{sym:functionvec}\left(\gls{sym:state}_{\gls{idx:ref}},\gls{sym:t}\right) + \left(\frac{\partial \gls{sym:functionvec}\left(\gls{sym:state},\gls{sym:t}\right)}{\partial
\xt}\right)_{\gls{idx:ref}} \cdot \left(\xt - \xref\right) + \mathcal{O}\left(\Delta\xt^2\right), \\
\yt &= \gls{sym:functiongvec}\left(\gls{sym:state}_{\gls{idx:ref}},\gls{sym:t}\right) + \left(\frac{\partial \gls{sym:functiongvec}\left(\gls{sym:state},\gls{sym:t}\right)}{\partial
\xt}\right)_{\gls{idx:ref}} \cdot \left(\xt - \xref\right) + \mathcal{O}\left(\Delta\xt^2\right) + \boldsymbol{\gls{sym:obsError}},
\end{align}
or, combined with \eq{eq:deviation-ref} and \eq{eq:deviation-ref-y} and neglecting higher-order terms:
\begin{align}
 \Delta \xdot &= \left(\frac{\partial \xdot}{\partial \xt}\right)_{\gls{idx:ref}} \Delta \xt, \\
 \Delta \yt &= \left(\frac{\partial \gls{sym:functiongvec}\left(\gls{sym:state},\gls{sym:t}\right)}{\partial \xt}\right)_{\gls{idx:ref}} \Delta \xt + \gls{sym:obsError}.
\end{align}
Defining the system matrix $\gls{sym:systemMat}\left(\gls{sym:t}\right)$ as
\begin{equation}
 \gls{sym:systemMat}\left(\gls{sym:t}\right) \equiv \left(\frac{\partial \xdot}{\partial \xt}\right)_{\gls{idx:ref}}
\end{equation}
and the output matrix as
\begin{equation}
 \gls{sym:outputMat}\left(\gls{sym:t}\right) \equiv \left(\frac{\partial \gls{sym:functiongvec}\left(\gls{sym:state},\gls{sym:t}\right)}{\partial \xt}\right)_{\gls{idx:ref}}
\end{equation}
the homogeneous part of the state space representation is obtained for the linearized system:
\begin{align}
 \Delta \xdot &= \gls{sym:systemMat}\left(\gls{sym:t}\right) \Delta \xt, \label{eq:diff-homo} \\
 \Delta \yt   &= \gls{sym:outputMat}\left(\gls{sym:t}\right) \Delta \xt + \gls{sym:obsError}. \label{eq:meas-model-linearized}
\end{align}
The general solution for \eq{eq:diff-homo} is:
\begin{align}
 \Delta \xt &= \frac{\partial \Delta \xt}{\partial \Delta\gls{sym:state}_0} \Delta \gls{sym:state}_0 \\
            &= \frac{\partial \left(\xt - \xref\right)}{\partial \left(\gls{sym:state}_0 - \gls{sym:state}_{\gls{idx:ref},0}\right)} \Delta \gls{sym:state}_0 
\label{eq:ref-constant}\\
            &= \frac{\partial \xt}{\partial \gls{sym:state}_0} \Delta \gls{sym:state}_0 \label{eq:set-homo}
\end{align}
given the initial condition $\gls{sym:state}\left(\gls{sym:t}_0\right)=\gls{sym:state}_0$ and noting that the reference state is constant in the partial derivative in
\eq{eq:ref-constant}. Now the state error transition matrix \gls{sym:set} can be introduced, which translates the state error $\Delta \xt$ from $\gls{sym:t}_0$ to $\gls{sym:t}$:
\begin{equation}
 \set{\gls{sym:t}}{\gls{sym:t}_0} \equiv \frac{\partial \xt}{\partial \gls{sym:state}_0}
\end{equation}
Differentiating \eq{eq:set-homo} provides
\begin{equation}
 \Delta \xdot = \dot{\gls{sym:set}}\left(\gls{sym:t},\gls{sym:t}_0\right) \Delta \gls{sym:state}_0,
\end{equation}
which, together with \eq{eq:set-homo}, can be substituted into \eq{eq:diff-homo}:
\begin{equation}
 \dot{\gls{sym:set}}\left(\gls{sym:t},\gls{sym:t}_0\right) \Delta \gls{sym:state}_0 = \gls{sym:systemMat}\left(\gls{sym:t}\right)
\set{\gls{sym:t}}{\gls{sym:t}_0} \Delta \gls{sym:state}_0,
\end{equation}
finally providing a differential equation for the state error transition matrix, which will later be integrated numerically for the propagation of the covariance matrix:
\begin{equation}
 \dot{\gls{sym:set}}\left(\gls{sym:t},\gls{sym:t}_0\right) = \gls{sym:systemMat}\left(\gls{sym:t}\right) \set{\gls{sym:t}}{\gls{sym:t}_0} \label{eq:set-ode}
\end{equation}

The main advantage of solving for the state error transition matrix using \eq{eq:set-ode} over the direct solution of \eq{eq:diff-homo} 
is that the state error transition matrix allows for a simple formulation of the covariance matrix propagation and the determination of 
the best estimate of the state vector \citep{tapley2004}.

The influence of system or process noise, which is characterized by the unmodeled accelerations, is provided by another term leading to the general formulation of the state
space representation:
\begin{equation}
 \Delta \xdot = \gls{sym:systemMat}\left(\gls{sym:t}\right) \Delta \xt + \gls{sym:inputMat} \left(\gls{sym:t}\right) \gls{sym:inputVec}\left(\gls{sym:t}\right),
\label{eq:state-space}
\end{equation}
here, $\gls{sym:inputVec}\left(\gls{sym:t}\right)$ is the process noise, while $\gls{sym:inputMat}\left(\gls{sym:t}\right)$ is the input matrix, which converts the unmodeled
accelerations into the quantities of the state vector.

A particular solution to \eq{eq:state-space} can be found via the \textit{variation of constants} method, starting with:
\begin{equation}
 \Delta \xt = \set{\gls{sym:t}}{\gls{sym:t}_0} \gls{sym:vop}\left(\gls{sym:t}\right). \label{eq:vop}
\end{equation}
Differentiating:
\begin{equation}
 \Delta \xdot = \dot{\gls{sym:set}}\left(\gls{sym:t},\gls{sym:t}_0\right) \gls{sym:vop}\left(\gls{sym:t}\right) + \set{\gls{sym:t}}{\gls{sym:t}_0}
\dot{\gls{sym:vop}}\left(\gls{sym:t}\right),
\end{equation}
and substituting into \eq{eq:state-space} provides
\begin{equation}
 \dot{\gls{sym:set}}\left(\gls{sym:t},\gls{sym:t}_0\right) \gls{sym:vop}\left(\gls{sym:t}\right) + \set{\gls{sym:t}}{\gls{sym:t}_0}
\dot{\gls{sym:vop}}\left(\gls{sym:t}\right) = \gls{sym:systemMat}\left(\gls{sym:t}\right) \Delta \xt + \gls{sym:inputMat} \left(\gls{sym:t}\right)
\gls{sym:inputVec}\left(\gls{sym:t}\right).
\end{equation}
This equation can be re-written by substituting the already known relationships for the time derivative of the state error transition matrix, \eq{eq:set-ode},
and \eq{eq:vop}: 
\begin{equation}
 \gls{sym:systemMat}\left(\gls{sym:t}\right) \set{\gls{sym:t}}{\gls{sym:t}_0} \gls{sym:vop}\left(\gls{sym:t}\right) + \set{\gls{sym:t}}{\gls{sym:t}_0}
\dot{\gls{sym:vop}}\left(\gls{sym:t}\right) = \gls{sym:systemMat}\left(\gls{sym:t}\right) \set{\gls{sym:t}}{\gls{sym:t}_0} \gls{sym:vop}\left(\gls{sym:t}\right)  +
\gls{sym:inputMat} \left(\gls{sym:t}\right) \gls{sym:inputVec}\left(\gls{sym:t}\right),
\end{equation}
which results in
\begin{equation}
 \set{\gls{sym:t}}{\gls{sym:t}_0} \dot{\gls{sym:vop}}\left(\gls{sym:t}\right) = \gls{sym:inputMat} \left(\gls{sym:t}\right) \gls{sym:inputVec}\left(\gls{sym:t}\right).
\end{equation}
The solution is now obtained by integration:
\begin{equation}
 \gls{sym:vop}\left(\gls{sym:t}\right) = \gls{sym:vop}_0 + \int_{\gls{sym:t}_0}^{\gls{sym:t}} \gls{sym:set}^{-1}\left(\gls{sym:xi},\gls{sym:t}_0\right)
\gls{sym:inputMat}\left(\gls{sym:xi}\right)\gls{sym:inputVec}\left(\gls{sym:xi}\right) d\gls{sym:xi}. \label{eq:result-ct}
\end{equation}

Substituting the result for $\gls{sym:vop}\left(\gls{sym:t}\right)$ from \eq{eq:result-ct} into \eq{eq:vop} results in:
\begin{equation}
 \Delta \xt = \set{\gls{sym:t}}{\gls{sym:t}_0} \gls{sym:vop}_0 + \int_{\gls{sym:t}_0}^{\gls{sym:t}} \set{\gls{sym:t}}{\gls{sym:t}_0}
\gls{sym:set}^{-1}\left(\gls{sym:xi},\gls{sym:t}_0\right)
\gls{sym:inputMat}\left(\gls{sym:xi}\right)\gls{sym:inputVec}\left(\gls{sym:xi}\right) d\gls{sym:xi},
\end{equation}
which can be simplified using the following properties of the state error transition matrix:
\begin{equation}
 \set{\gls{sym:t}}{\gls{sym:t}_0}\gls{sym:set}^{-1}\left(\gls{sym:xi},\gls{sym:t}_0\right) = \set{\gls{sym:t}}{\gls{sym:t}_0} \set{\gls{sym:t}_0}{\gls{sym:xi}}
 = \set{\gls{sym:t}}{\gls{sym:xi}},
\end{equation}
as well as the initial condition $\gls{sym:vop}_0 = \gls{sym:state}_0$ to finally provide the general solution for the inhomogeneous \eq{eq:state-space}:
\begin{equation}
 \Delta \xt = \set{\gls{sym:t}}{\gls{sym:t}_0} \Delta \gls{sym:state}_0 + \int_{\gls{sym:t}_0}^{\gls{sym:t}} \set{\gls{sym:t}}{\gls{sym:xi}}
\gls{sym:inputMat}\left(\gls{sym:xi}\right)\gls{sym:inputVec}\left(\gls{sym:xi}\right) d\gls{sym:xi}. \label{eq:state-space-general-solution}
\end{equation}

The result in \eq{eq:state-space-general-solution} is also referred to as the \textit{matrix superposition integral} \citep{gelb1974}, where the second term describes how an input 
(here: process noise \gls{sym:inputVec}) at a time \gls{sym:xi} translates into a state vector change at time \gls{sym:t}.

%------------------------------------------------------------------------------
\subsection{General formulation using the state transition matrix}
\label{sec:propagation-covariance-theory}
%------------------------------------------------------------------------------

Using the state space representation, as introduced in \sect{sec:state-space-notation}, with its general solution given by \eq{eq:state-space-general-solution},
 the covariance matrix at a time \gls{sym:t} follows as:
\begin{alignat}{3}
 \gls{sym:P}\left(\gls{sym:t}\right) = \gls{sym:expVal}\left[\Delta\gls{sym:state}\left(\gls{sym:t}\right)\cdot\Delta\gls{sym:state}\left(\gls{sym:t}
\right)^T\right]&=
\gls{sym:expVal}\Bigl[&&\left(\gls{sym:set}\left(\gls{sym:t},\gls{sym:t}_0\right)\cdot
\Delta\gls{sym:state}\left(\gls{sym:t}_0\right)+
\int_{\gls{sym:t}_0}^{\gls{sym:t}}\gls{sym:set}\left(\gls{sym:t},\gls{sym:xi}\right)\cdot
\gls{sym:inputMat}\left(\gls{sym:xi}\right)\cdot\gls{sym:inputVec}\left(\gls{sym:xi}\right) d\gls{sym:xi}\right)\times \notag \\
& &&\left(\gls{sym:set}\left(\gls{sym:t},\gls{sym:t}_0\right)\cdot
\Delta\gls{sym:state}\left(\gls{sym:t}_0\right)+ \int_{\gls{sym:t}_0}^{\gls{sym:t}}\gls{sym:set}\left(\gls{sym:t},\gls{sym:eta}\right)\cdot
\gls{sym:inputMat}\left(\gls{sym:eta}\right)\cdot\gls{sym:inputVec}\left(\gls{sym:eta}\right) d\gls{sym:eta}\right)^T\Bigr] \notag \\
&= &&\gls{sym:set}\left(\gls{sym:t},\gls{sym:t}_0\right)\cdot\gls{sym:expVal}\left[\Delta\gls{sym:state}\left(\gls{sym:t}_0\right)\cdot
\Delta\gls{sym:state}\left(\gls{sym:t}_0\right)^T\right]\cdot \gls{sym:set}\left(\gls{sym:t},\gls{sym:t}_0\right)^T + \label{eq:full-cov-prop} \\
& &&
\gls{sym:set}\left(\gls{sym:t},\gls{sym:t}_0\right)\cdot\int_{\gls{sym:t}_0}^{\gls{sym:t}}\gls{sym:expVal}\left[\Delta\gls{sym:state}\left(\gls{sym:t}_0\right)
\cdot\gls{sym:inputVec}\left(\gls{sym:eta}\right)^T\right]\cdot\gls{sym:inputMat}\left(\gls{sym:eta}\right)\cdot
\gls{sym:set}\left(\gls{sym:t},\gls{sym:eta}\right) d\gls{sym:eta}+ \notag \\
& &&
\left(\gls{sym:set}\left(\gls{sym:t},\gls{sym:t}_0\right)\cdot\int_{\gls{sym:t}_0}^{\gls{sym:t}}\gls{sym:expVal}\left[\Delta\gls{sym:state}\left(\gls{sym:t}_0\right)
\cdot\gls{sym:inputVec}\left(\gls{sym:eta}\right)^T\right]\cdot\gls{sym:inputMat}\left(\gls{sym:eta}\right)\cdot
\gls{sym:set}\left(\gls{sym:t},\gls{sym:eta}\right) d\gls{sym:eta}\right)^T + \notag \\
& &&
\int_{\gls{sym:t}_0}^{\gls{sym:t}} \int_{\gls{sym:t}_0}^{\gls{sym:t}} \gls{sym:set}\left(\gls{sym:t},\gls{sym:xi}\right) \cdot
\gls{sym:inputMat}\left(\gls{sym:xi}\right) \cdot
\gls{sym:expVal}\left[\gls{sym:inputVec}\left(\gls{sym:xi}\right)\cdot \gls{sym:inputVec}\left(\gls{sym:eta}\right)^T\right]
\gls{sym:inputMat}\left(\gls{sym:eta}\right)^T \cdot
\gls{sym:set}\left(\gls{sym:t},\gls{sym:eta}\right)^T d\gls{sym:xi} d\gls{sym:eta}. \notag
\end{alignat}
Analysing the individual terms in the above sum, the first one is the time update of the covariance matrix from $\gls{sym:t}_0$ to \gls{sym:t}, with
\begin{equation}
  \gls{sym:P}\left(\gls{sym:t}_0\right) = \gls{sym:expVal}\left[\Delta\gls{sym:state}\left(\gls{sym:t}_0\right)\cdot\Delta\gls{sym:state}\left(\gls{sym:t}_0\right)^T\right ].
\end{equation}

The second and third term in \eq{eq:full-cov-prop} are similar in that the one is obtained from the other by transposing. That contribution results 
from the cross-correlation of the state vector at $\gls{sym:t}_0$ and the process noise and shall be defined as \gls{sym:Qxu}:
\begin{equation}
 \gls{sym:Qxu}\left(\gls{sym:t}\right) =
\gls{sym:set}\left(\gls{sym:t},\gls{sym:t}_0\right)\cdot\int_{\gls{sym:t}_0}^{\gls{sym:t}}\gls{sym:expVal}\left[\Delta\gls{sym:state}\left(\gls{sym:t}_0\right)
\cdot\gls{sym:inputVec}\left(\gls{sym:eta}\right)^T\right]\cdot\gls{sym:inputMat}\left(\gls{sym:eta}\right)\cdot
\gls{sym:set}\left(\gls{sym:t},\gls{sym:eta}\right) d\gls{sym:eta}
\end{equation}

The last term in \eq{eq:full-cov-prop} shall be denoted as $\gls{sym:Quu}\left(\gls{sym:t}\right)$, being the second moment of process noise matrix:
\begin{equation}
 \gls{sym:Quu}\left(\gls{sym:t}\right) = \int_{\gls{sym:t}_0}^{\gls{sym:t}} \int_{\gls{sym:t}_0}^{\gls{sym:t}} \gls{sym:set}\left(\gls{sym:t},\gls{sym:xi}\right) \cdot
\gls{sym:inputMat}\left(\gls{sym:xi}\right) \cdot
\gls{sym:expVal}\left[\gls{sym:inputVec}\left(\gls{sym:xi}\right)\cdot \gls{sym:inputVec}\left(\gls{sym:eta}\right)^T\right]
\gls{sym:inputMat}\left(\gls{sym:eta}\right)^T \cdot
\gls{sym:set}\left(\gls{sym:t},\gls{sym:eta}\right)^T d\gls{sym:xi} d\gls{sym:eta}.
\end{equation}

It is now possible to write \eq{eq:full-cov-prop} in a more condensed way:
\begin{equation}
 \gls{sym:P}\left(\gls{sym:t}\right) = \set{\gls{sym:t}}{\gls{sym:t}_0} \gls{sym:P}_0 \set{\gls{sym:t}}{\gls{sym:t}_0}^T + \gls{sym:Qxu}\left(\gls{sym:t}\right) +
\gls{sym:Qxu}\left(\gls{sym:t}\right)^T + \gls{sym:Quu}\left(\gls{sym:t}\right).
\end{equation}

%------------------------------------------------------------------------------
\subsection{SET matrix integration}
\label{sec:propagation-covariance-set-integration}
%------------------------------------------------------------------------------
The state vector \gls{sym:state} is defined as
\begin{equation}
 \gls{sym:state} = \left(\gls{sym:radvec}, \gls{sym:velvec}\right)^T 
                 = \left(\rx, \ry, \rz, \gls{sym:v}_{\gls{idx:x}}, \gls{sym:v}_{\gls{idx:y}}, 
\gls{sym:v}_{\gls{idx:z}}\right)^T
\end{equation}

The partial derivative matrix \gls{sym:matrixF}, without any solve-for parameters, is defined as
\begin{equation}
 \gls{sym:matrixF}\left(\gls{sym:tg}\right) = \frac{\partial \dot{\gls{sym:state}}}{\partial \gls{sym:state}} = 
 \begin{pmatrix}
    % first row
    \frac{\partial \gls{sym:v}_{\gls{idx:x}}}{\partial \rx} & 
    \frac{\partial \gls{sym:v}_{\gls{idx:x}}}{\partial \ry} &     
    \frac{\partial \gls{sym:v}_{\gls{idx:x}}}{\partial \rz} & 
    \frac{\partial \gls{sym:v}_{\gls{idx:x}}}{\partial \gls{sym:v}_{\gls{idx:x}}} &
    \frac{\partial \gls{sym:v}_{\gls{idx:x}}}{\partial \gls{sym:v}_{\gls{idx:y}}} &
    \frac{\partial \gls{sym:v}_{\gls{idx:x}}}{\partial \gls{sym:v}_{\gls{idx:z}}} \\
    % second row
    \frac{\partial \gls{sym:v}_{\gls{idx:y}}}{\partial \rx} &
    \frac{\partial \gls{sym:v}_{\gls{idx:y}}}{\partial \ry} & 
    \frac{\partial \gls{sym:v}_{\gls{idx:y}}}{\partial \rz} & 
    \frac{\partial \gls{sym:v}_{\gls{idx:y}}}{\partial \gls{sym:v}_{\gls{idx:x}}} &
    \frac{\partial \gls{sym:v}_{\gls{idx:y}}}{\partial \gls{sym:v}_{\gls{idx:y}}} & 
    \frac{\partial \gls{sym:v}_{\gls{idx:y}}}{\partial \gls{sym:v}_{\gls{idx:z}}} \\
    % third row
    \frac{\partial \gls{sym:v}_{\gls{idx:z}}}{\partial \rx} &
    \frac{\partial \gls{sym:v}_{\gls{idx:z}}}{\partial \ry} & 
    \frac{\partial \gls{sym:v}_{\gls{idx:z}}}{\partial \rz} &
    \frac{\partial \gls{sym:v}_{\gls{idx:z}}}{\partial \gls{sym:v}_{\gls{idx:x}}} &
    \frac{\partial \gls{sym:v}_{\gls{idx:z}}}{\partial \gls{sym:v}_{\gls{idx:y}}} &
    \frac{\partial \gls{sym:v}_{\gls{idx:z}}}{\partial \gls{sym:v}_{\gls{idx:z}}} \\
    % fourth row   
    \frac{\partial \gls{sym:a}_{\gls{idx:x}}}{\partial \rx} &
    \frac{\partial \gls{sym:a}_{\gls{idx:x}}}{\partial \ry} & 
    \frac{\partial \gls{sym:a}_{\gls{idx:x}}}{\partial \rz} &
    \frac{\partial \gls{sym:a}_{\gls{idx:x}}}{\partial \gls{sym:v}_{\gls{idx:x}}} &
    \frac{\partial \gls{sym:a}_{\gls{idx:x}}}{\partial \gls{sym:v}_{\gls{idx:y}}} &
    \frac{\partial \gls{sym:a}_{\gls{idx:x}}}{\partial \gls{sym:v}_{\gls{idx:z}}} \\
    % fifth row
    \frac{\partial \gls{sym:a}_{\gls{idx:y}}}{\partial \rx} &
    \frac{\partial \gls{sym:a}_{\gls{idx:y}}}{\partial \ry} & 
    \frac{\partial \gls{sym:a}_{\gls{idx:y}}}{\partial \rz} &
    \frac{\partial \gls{sym:a}_{\gls{idx:y}}}{\partial \gls{sym:v}_{\gls{idx:x}}} &
    \frac{\partial \gls{sym:a}_{\gls{idx:y}}}{\partial \gls{sym:v}_{\gls{idx:y}}} &
    \frac{\partial \gls{sym:a}_{\gls{idx:y}}}{\partial \gls{sym:v}_{\gls{idx:z}}} \\
    % sixth row
    \frac{\partial \gls{sym:a}_{\gls{idx:z}}}{\partial \rx} &
    \frac{\partial \gls{sym:a}_{\gls{idx:z}}}{\partial \ry} & 
    \frac{\partial \gls{sym:a}_{\gls{idx:z}}}{\partial \rz} &
    \frac{\partial \gls{sym:a}_{\gls{idx:z}}}{\partial \gls{sym:v}_{\gls{idx:x}}} &
    \frac{\partial \gls{sym:a}_{\gls{idx:z}}}{\partial \gls{sym:v}_{\gls{idx:y}}} &
    \frac{\partial \gls{sym:a}_{\gls{idx:z}}}{\partial \gls{sym:v}_{\gls{idx:z}}}    
  \end{pmatrix} \label{eq:partial-derivative-matrix}
\end{equation}

%------------------------------------------------------------------------------
\subsubsection{Partial derivatives for the geopotential}
\label{sec:propagation-covariance-set-integration-geopotential}
%------------------------------------------------------------------------------

The partial derivatives of the acceleration contribution due to the non-spherical Earth in the body fixed frame, are
determined according to \cite{long1989}. 
As they are only a function of the radius vector, the partial derivate matrix contribution of the geopotential results in (see 
\eq{eq:partial-derivative-matrix}):
\begin{equation}
 \gls{sym:matrixF}_{\gls{idx:geop}}\left(\gls{sym:tg}\right) =  
 \begin{pmatrix}
    % first row
    0 & 0 & 0 & 1 & 0 & 0 \\
    % second row
    0 & 0 & 0 & 0 & 1 & 0 \\
    % third row
    0 & 0 & 0 & 0 & 0 & 1 \\
    % fourth row   
    \frac{\partial \gls{sym:a}_{\gls{idx:x}}}{\partial \rx} &
    \frac{\partial \gls{sym:a}_{\gls{idx:x}}}{\partial \ry} & 
    \frac{\partial \gls{sym:a}_{\gls{idx:x}}}{\partial \rz} &
    0 & 0 & 0 \\
    % fifth row
    \frac{\partial \gls{sym:a}_{\gls{idx:y}}}{\partial \rx} &
    \frac{\partial \gls{sym:a}_{\gls{idx:y}}}{\partial \ry} & 
    \frac{\partial \gls{sym:a}_{\gls{idx:y}}}{\partial \rz} &
    0 & 0 & 0 \\
    % sixth row
    \frac{\partial \gls{sym:a}_{\gls{idx:z}}}{\partial \rx} &
    \frac{\partial \gls{sym:a}_{\gls{idx:z}}}{\partial \ry} & 
    \frac{\partial \gls{sym:a}_{\gls{idx:z}}}{\partial \rz} &
    0 & 0 & 0    
  \end{pmatrix} \label{eq:partial-derivative-matrix-geopotential}
\end{equation}
The individual derivatives are based on a differentiation of the acceleration vector components derived for the geopotential 
in \eq{eq:geopotential-acceleration-1} through \eq{eq:geopotential-acceleration-2} and are computed as follows:
\begin{equation}
  \begin{aligned}
 \frac{\partial \gls{sym:avec}_{\gls{idx:geop},\gls{idx:bodyfixed}}}{\partial \rbf} &= 
  \frac{\partial}{\partial\rbf} \left(\frac{\partial\gls{sym:potential}}{\partial\gls{sym:r}}\right) 
  \frac{\partial\gls{sym:radvec}}{\partial\rbf} +
  \frac{\partial}{\partial\rbf} \left(\frac{\partial\gls{sym:potential}}{\partial\phigc}\right) 
  \frac{\partial\phigc}{\partial\rbf} +
  \frac{\partial}{\partial\rbf} \left(\frac{\partial\gls{sym:potential}}{\partial\gls{sym:long}}\right) 
  \frac{\partial\gls{sym:long}}{\partial\rbf}  \\
  &+ \frac{\partial\gls{sym:potential}}{\partial\gls{sym:r}} \frac{\partial^2\gls{sym:r}}{\partial\rbf} +
      \frac{\partial\gls{sym:potential}}{\partial\phigc} \frac{\partial^2\gls{sym:lat}}{\partial\rbf} +
      \frac{\partial\gls{sym:potential}}{\partial\gls{sym:long}} \frac{\partial^2\gls{sym:long}}{\partial\rbf} 
\label{eq:geop-acc-derivative}
\end{aligned}
\end{equation}

The partial derivatives of the first three terms in \eq{eq:geop-acc-derivative}, $\partial\gls{sym:potential}/\partial\gls{sym:r}$, 
$\partial\gls{sym:potential}/\partial\phigc$, $\partial\gls{sym:potential}/\partial\gls{sym:long}$, with respect to the
body-fixed radius vector, $\gls{sym:r}_{\gls{idx:bodyfixed}}$, are obtained by differentiating
\eq{eq:geopotential-acceleration-1} through \eq{eq:geopotential-acceleration-2} 
using the following notation \citep{long1989}:
\begin{equation}
 \frac{\partial}{\partial\rbf}
 \begin{bmatrix}
  \frac{\partial\gls{sym:potential}}{\partial\gls{sym:r}} \\[0.5em]
  \frac{\partial\gls{sym:potential}}{\partial\phigc} \\[0.5em]
  \frac{\partial\gls{sym:potential}}{\partial\gls{sym:long}}
 \end{bmatrix} 
 = 
 \begin{pmatrix}
  %first row
  \frac{\partial^2\gls{sym:potential}}{\partial\gls{sym:r}^2} & 
  \frac{\partial^2\gls{sym:potential}}{\partial\gls{sym:r}\partial\phigc} & 
  \frac{\partial^2\gls{sym:potential}}{\partial\gls{sym:r}\partial\gls{sym:long}} \\
  % second row
  \frac{\partial^2\gls{sym:potential}}{\partial\phigc\partial\gls{sym:r}} & 
  \frac{\partial^2\gls{sym:potential}}{\partial\gls{sym:lat}^2_{\gls{idx:gc}}} & 
  \frac{\partial^2\gls{sym:potential}}{\partial\phigc\partial\gls{sym:long}} \\
  % third row
  \frac{\partial^2\gls{sym:potential}}{\partial\gls{sym:long}\partial\gls{sym:r}} &
  \frac{\partial^2\gls{sym:potential}}{\partial\gls{sym:long}\partial\gls{sym:r}} &
  \frac{\partial^2\gls{sym:potential}}{\partial\gls{sym:long}\partial\phigc}
 \end{pmatrix}
 \begin{bmatrix}
  \frac{\partial\gls{sym:radvec}}{\partial\gls{sym:r}_{\gls{idx:bodyfixed}}} \\[0.5em]
  \frac{\partial\phigc}{\partial\rbf} \\[0.5em]
  \frac{\partial\gls{sym:long}}{\partial\rbf}
 \end{bmatrix} \label{eq:second-derivative-potential} 
\end{equation}
For example, the first element in \eq{eq:second-derivative-potential} would be read as:
\begin{equation}
 \frac{\partial}{\partial\rbf} \left(\frac{\partial\gls{sym:potential}}{\partial\gls{sym:r}}\right) =
 \frac{\partial^2\gls{sym:potential}}{\partial\gls{sym:r}^2} \frac{\partial\gls{sym:r}}{\partial\rbf} +
 \frac{\partial^2\gls{sym:potential}}{\partial\gls{sym:r}\partial\phigc} 
\frac{\partial\phigc}{\partial\rbf} +
\frac{\partial^2\gls{sym:potential}}{\partial\gls{sym:r}\partial\gls{sym:long}} \frac{\partial\gls{sym:long}}{\partial\rbf}
\end{equation}
The matrix in \eq{eq:second-derivative-potential} is symmetric, so that only six different second derivatives have to be computed, based on the first 
derivatives in \eq{eq:deriv-geopotential}, as well as the derivatives of the associated Legendre functions (\eq{eq:deriv-legendre-phi} and 
\eq{eq:deriv-legendre-phi-2}) with respect to the  independent variable (being the sine of the geocentric latitude):
\begin{align}
%
% d^2U/dr^2
%
 \frac{\partial^2\gls{sym:potential}}{\partial\gls{sym:r}^2} = &\frac{\gls{sym:grav}}{\gls{sym:r}^3} 
             \sum\limits_{\gls{sym:geo_n}=2}^{\infty} \left(\frac{\re}{\gls{sym:r}}\right)^{\gls{sym:geo_n}} 
\left(\gls{sym:geo_n}+2\right)\left(\gls{sym:geo_n}+1\right) \times \notag \\
      &\qquad \qquad \times \sum\limits_{\gls{sym:geo_m}=0}^{\gls{sym:geo_n}} \left(\cnm{}{} \cos 
\left(\gls{sym:geo_m}\gls{sym:long}\right) + \snm{}{} \sin \left(\gls{sym:geo_m}\gls{sym:long}\right)\right) 
\legphi{}{} \displaybreak[0] \\[1em]
%
% d^2U/drdphi
%
 \frac{\partial^2\gls{sym:potential}}{\partial\gls{sym:r}\partial\phigc} = &
 \frac{\partial^2\gls{sym:potential}}{\partial\phigc\partial\gls{sym:r}} = -\frac{\gls{sym:grav}}{\gls{sym:r}^2} 
             \sum\limits_{\gls{sym:geo_n}=2}^{\infty} 
\left(\frac{\re}{\gls{sym:r}}\right)^{\gls{sym:geo_n}} \left(\gls{sym:geo_n}+1\right) 
\sum\limits_{\gls{sym:geo_m}=0}^{\gls{sym:geo_n}} \left(\cnm{}{} \cos 
\left(\gls{sym:geo_m}\gls{sym:long}\right) + \snm{}{} \sin \left(\gls{sym:geo_m}\gls{sym:long}\right)\right) \times \notag 
\\
 & \qquad \qquad \times \left(\legphi{}{+1} - \gls{sym:geo_m} \tan \phigc \legphi{}{} \right) \displaybreak[0] \\[1em]
%
% d^2U/drdlambda
%
 \frac{\partial^2\gls{sym:potential}}{\partial\gls{sym:r}\partial\gls{sym:long}} = & 
 \frac{\partial^2\gls{sym:potential}}{\partial\gls{sym:long}\partial\gls{sym:r}} = -\frac{\gls{sym:grav}}{\gls{sym:r}^2} 
             \sum\limits_{\gls{sym:geo_n}=2}^{\infty} 
\left(\frac{\re}{\gls{sym:r}}\right)^{\gls{sym:geo_n}} \left(\gls{sym:geo_n}+1\right) 
\sum\limits_{\gls{sym:geo_m}=0}^{\gls{sym:geo_n}} \gls{sym:geo_m} \left(\snm{}{} \cos 
\left(\gls{sym:geo_m}\gls{sym:long}\right) - \cnm{}{} \sin \left(\gls{sym:geo_m}\gls{sym:long}\right)\right) \times \notag 
\\
 & \qquad \qquad \times \legphi{}{} \displaybreak[0] \\[1em]
%
% d^2U/dphi^2
%
 \frac{\partial^2\gls{sym:potential}}{\partial\phigc^2} = &\frac{\gls{sym:grav}}{\gls{sym:r}} 
 \sum\limits_{\gls{sym:geo_n}=2}^{\infty} \left(\frac{\re}{\gls{sym:r}}\right)^{\gls{sym:geo_n}} \sum\limits_{\gls{sym:geo_m}=0}^{\gls{sym:geo_n}} 
 \left(\cnm{}{} \cos \left(\gls{sym:geo_m}\gls{sym:long}\right) + \snm{}{} \sin \left(\gls{sym:geo_m}\gls{sym:long}\right)\right) 
\left(\tan\phigc\legphi{}{+1} \right. + \notag \\  & \qquad \qquad + \left. \left(\gls{sym:geo_m}^2\sec^2\phigc - \gls{sym:geo_m}\tan^2\phigc - 
\gls{sym:geo_n}\left(\gls{sym:geo_n}+1\right)\right)\legphi{}{}\right) \displaybreak[0]  \label{eq:d2Udphi2}\\[1em]
%
% d^2U/dphidlambda
 \frac{\partial^2\gls{sym:potential}}{\partial\phigc\partial\gls{sym:long}} = & 
 \frac{\partial^2\gls{sym:potential}}{\partial\gls{sym:long}\partial\phigc} = \frac{\gls{sym:grav}}{\gls{sym:r}} \sum\limits_{\gls{sym:geo_n}=2}^{\infty} 
\left(\frac{\re}{\gls{sym:r}}\right)^{\gls{sym:geo_n}} \sum\limits_{\gls{sym:geo_m}=0}^{\gls{sym:geo_n}} \gls{sym:geo_m} \left(\snm{}{} \cos 
\left(\gls{sym:geo_m}\gls{sym:long}\right) - \cnm{}{} \sin \left(\gls{sym:geo_m}\gls{sym:long}\right)\right) \times \notag \\
 & \qquad \qquad \times \left(\legphi{}{+1} - \gls{sym:geo_m} \tan \phigc \legphi{}{} \right) \displaybreak[0] \\[1em]
%
% d^2U/dlambda^2
%
 \frac{\partial^2\gls{sym:potential}}{\partial\gls{sym:long}^2} = &
 -\frac{\gls{sym:grav}}{\gls{sym:r}^2} \sum\limits_{\gls{sym:geo_n}=2}^{\infty} \left(\frac{\re}{\gls{sym:r}}\right)^{\gls{sym:geo_n}} 
\sum\limits_{\gls{sym:geo_m}=0}^{\gls{sym:geo_n}} \gls{sym:geo_m}^2 \left(\cnm{}{} \cos \left(\gls{sym:geo_m}\gls{sym:long}\right) + \snm{}{} \sin 
\left(\gls{sym:geo_m}\gls{sym:long}\right)\right) \legphi{}{}  
\end{align}
Note that there is no discontinuity for $\phigc=90^\circ$ in \eq{eq:d2Udphi2}, as for $\gls{sym:geo_m}=0$ the tangent is reduced by a cosine function provided 
by the \legphi{}{+1} function and for $\gls{sym:geo_m}=1$ one obtains for the combination of secant and tangent:
\begin{equation}
 \sec^2\phigc - \tan^2\phigc = 1
\end{equation}
Finally, for all $\gls{sym:geo_m}>1$, there is always a $\cos^2\phigc$ term, which reduces the secant (to unity) and the squared tangent (to $\sin^2\phigc$).

The second part of \eq{eq:geop-acc-derivative} contains second derivatives of \gls{sym:r}, \phigc and \gls{sym:long} with respect to the radius in the 
body-fixed frame. These derivatives are obtained by differentiating \eqsto{eq:deriv-spherical-1}{eq:deriv-spherical-3} \citep{long1989}:
\begin{align}
 \frac{\partial^2\gls{sym:r}}{\partial\rbf^2}    =& \frac{1}{\gls{sym:r}}\left(\gls{sym:identity}-\frac{\rbf\rbf^T}{\gls{sym:r}^2}\right) \\[1em]
 \frac{\partial^2\phigc}{\partial\rbf^2}         =& 
-\frac{1}{\left(\rx^2+\ry^2\right)^{3/2}}\left(\left(\frac{\partial\rz}{\partial\rbf}\right)^T-\frac{\rz\rbf}{\gls{sym:r}^2}
\right)\left(\rx\left(\frac{\partial\rx}{\partial\rbf}\right)+\ry\left(\frac{\partial\ry}{\partial\rbf}\right)\right) - \notag \\[1em]
 & 
-\frac{1}{\gls{sym:r}^2\sqrt{\rx^2+\ry^2}}\left(\rbf\left(\frac{\partial\rz}{\partial\rbf}\right)+\rz\gls{sym:identity}-\frac{2\rz}{\gls{sym:r}^2}
\rbf\rbf^T\right) \\[1em]
 \frac{\partial^2\gls{sym:long}}{\partial\rbf^2} =& \frac{1}{\left(\rx^2+\ry^2\right)^2}
 \begin{bmatrix}
 -\ry \\
 \rx  \\
  0
 \end{bmatrix}
 \left(\rx\left(\frac{\partial\rx}{\partial\rbf}\right)+\ry\left(\frac{\partial\ry}{\partial\rbf}\right)\right) +
 \frac{1}{\rx^2+\ry^2}
 \begin{bmatrix}
  0 & -1 & 0 \\
  1 & 0  & 0 \\
  0 & 0  & 0 
 \end{bmatrix}
\end{align}
While the above equations appear to be quite complex, they can be significantly simplified, considering the fact that the partial derivatives of the radius 
components are unit vectors:
\begin{align}
 \frac{\partial\rx}{\partial\rbf} = \left[1,0,0\right] \notag \\
 \frac{\partial\ry}{\partial\rbf} = \left[0,1,0\right] \notag \\
 \frac{\partial\rz}{\partial\rbf} = \left[0,0,1\right] \notag 
\end{align}

Now all quantities are available to compute the inertial acceleration $\gls{sym:avec}_{\gls{idx:bodyfixed}}$ in the body-fixed (\acrshort{acr:itrf}) frame. 
In order to obtain the derivative of the acceleration in the \acrshort{acr:gcrf} with respect to the radius vector in the \acrshort{acr:gcrf},  
the relationship from \eq{eq:ns-trafo} is used:
\begin{equation}
 \frac{\partial\gls{sym:avec}_{\gls{idx:ns}}}{\partial\gls{sym:radvec}} = 
\kot{itrf}{gcrf}\frac{\partial\gls{sym:avec}_{\gls{idx:bodyfixed}}}{\partial\rbf}\frac{\partial\rbf}{\partial\gls{sym:radvec}}= 
\kot{itrf}{gcrf}\frac{\partial\gls{sym:avec}_{\gls{idx:bodyfixed}}}{\partial\rbf}\left(\kot{itrf}{gcrf}\right)^T
\end{equation}

%------------------------------------------------------------------------------
\subsubsection{Partial derivatives for atmospheric drag contributions}
\label{sec:propagation-covariance-set-integration-drag}
%------------------------------------------------------------------------------

The computation of partial derivatives due to atmospheric drag is based on \eq{eq:drag-acceleration}, the latter being a function of the relative velocity, 
which deviates from the inertial velocity due to the rotating atmosphere and horizontal wind. A similar procedure to the one shown in the following was 
presented in \cite{long1989}. Provided that the relative velocity is already available in the \gls{acr:gcrf}, the derivatives of the acceleration vector with 
respect to the inertial velocity vector can be determined as follows:
\begin{equation}
 \frac{\partial\gls{sym:avec}_{\gls{idx:drag}}}{\partial\gls{sym:velvec}} = 
   \frac{\partial\gls{sym:avec}_{\gls{idx:drag}}}{\partial\relvel}\frac{\partial\relvel}{\partial\gls{sym:velvec}} =
   \frac{\partial\gls{sym:avec}_{\gls{idx:drag}}}{\partial\relvel} = 
-\frac{\density}{2}\frac{\cd\crossSection}{\gls{sym:mass}}\left(\gls{sym:identity}\left|\relvel\right|+\frac{\relvel\relvel^T}{\left|\relvel\right|}\right).
\end{equation}
The partial derivative of the relative velocity with respect to the inertial velocity is one, which is due to the definition of the relative 
velocity in the \gls{acr:gcrf}:
\begin{equation}
 \relvel = \gls{sym:velvec} + f\left(\gls{sym:radvec},t\right)
 \label{eq:relvel-fr}
\end{equation}
The additional terms, designated with $f\left(\gls{sym:radvec},t\right)$ are due to the already mentioned rotating atmosphere as well as horizontal wind and 
are only functions of the radius vector and time.

Neglecting horizontal wind contributions, the relative velocity can be given as a function of the rotating atmosphere alone, being assumed to co-rotate with 
the Earth, resulting in:
\begin{equation}
 \relvel = \gls{sym:velvec} - \gls{sym:rot}_{,\gls{acr:gcrf}}\times\gls{sym:radvec}
 \label{eq:relvel-rot}
\end{equation}

The angular rotation vector of the Earth, \gls{sym:rot}, is defined in the \gls{acr:tirs} as 
$\gls{sym:rot}_{,\gls{acr:tirs}}=\left[0,0,\omega_{\gls{idx:earth}}\right]^T$. As the relative velocity is evaluated in the \gls{acr:gcrf}, one has to take care 
of the appropriate frame transformation. 

The associated partial derivatives of the acceleration vector with respect to the radius vector thus result as:
\begin{equation}
 \frac{\partial\gls{sym:avec}_{\gls{idx:drag}}}{\partial\gls{sym:radvec}} = 
   \frac{\partial\gls{sym:avec}_{\gls{idx:drag}}}{\partial\relvel}\frac{\partial\relvel}{\partial\gls{sym:radvec}} =
  -\frac{\partial\gls{sym:avec}_{\gls{idx:drag}}}{\partial\relvel}\left(\kot{tirs}{gcrf} \gls{sym:rot}_{,\gls{acr:tirs}}\right)
\end{equation}
Making use of the fact, that $\gls{sym:rot}_{,\gls{acr:tirs}}$ has only a non-zero component in \gls{sym:zg}-direction, the first rotation in the 
transformation from \gls{acr:tirs} to \gls{acr:gcrf} (reduction of the \acrfull{acr:era}, see \eq{eq:ERA}) is not required (identity).