%------------------------------------------------------------------------------
\section{Reference frames}
\label{sec:propagation-coordinates}
%------------------------------------------------------------------------------

In orbit propagation, the use of different coordinate and time systems is unavoidable, which is mainly because the equations of motion are most easily solved in an 
inertial frame, whereas force models or observations are applied with respect to non-inertial frames, for example a co-rotating Earth-fixed frame.

Therefore, it is important to deal with the required coordinate and time systems at an early stage of the development process. As the orbit propagator is supposed to 
receive input states related to a certain coordinate frame and outputs the result again with respect to a certain frame, it is only to be recommended to use 
standardised methods at this point.

Based on the resolutions of the \gls{acr:iau} and \gls{acr:iugg} the \gls{acr:iers} realised and defined a celestial and a terrestrial reference frame \citep{luzum2010}. 
It is important to differentiate between \textit{reference systems} and \textit{reference frames}. While a reference system is the conceptual definition of a 
coordinate system, a reference frame is the actual realisation, using observations, station coordinates, etc. \citep{seidelmann2006}.

The celestial system, named \gls{acr:icrs},  was defined by the \gls{acr:iau} Resolution A4 (1991) and was refined in 2000 and 2009 \citep{luzum2010}. Requiring a geocentric 
reference system, the \gls{acr:gcrs} is a system oriented according to the \gls{acr:icrs} axes \citep{luzum2010}.

The terrestrial system, named \gls{acr:itrs}, is based on \gls{acr:iugg} Resolution 2 (1991) \citep{luzum2010}. The transformation between these two systems accounts for three 
different effects:
\begin{itemize}
 \item The motion of the celestial pole with respect to the celestial reference system (\textit{precession} of the ecliptic and the equator, as well as \textit{nutation}),
 \item the rotation of the Earth and
 \item the polar motion.
\end{itemize}
The transformations are typically referred to as the \gls{acr:cio} approach, which was defined by the \acrshort{acr:iau}-2006/2000 resolutions \citep{vallado2013}. In particular, 
this means that the \acrshort{acr:iau}-2000 Nutation theory and the \acrshort{acr:iau}-2006 Precession, the latter based on the \textit{P03 model} \citep{capitaine2003, wallace2006}, 
are the recommended models. The conversion uncertainty is on the order of milliarcseconds (mas), with 1 mas corresponding to a displacement of \SI{3}{\centi\metre} at
a distance of one Earth radius \citep{coppola2009}.

The realisation of the \gls{acr:itrs}, denoted as \gls{acr:itrf}, is regularly revised, with the axes definitions based on a weighted combination of a varying number of precisely
known station coordinates. Since 1984, thirteen versions have been published, from \acrshort{acr:itrf}88 to \acrshort{acr:itrf}2014. Using the \gls{acr:eop}
provided and regularly updated by the \gls{acr:iers} for the current realisation, \acrshort{acr:itrf}2014, this frame will in the following be referenced without its suffix:
\acrshort{acr:itrf}.

The realisation of the \gls{acr:icrs}, denoted as \gls{acr:icrf}, is defined for the barycenter of the solar system. For Earth-bound orbits it is advantageous to use a 
similar realisation, called \gls{acr:gcrf}, which has the same orientation as the \gls{acr:icrf}, but has its origin at the center of mass of the Earth. The 
axes are realised from \gls{acr:vlbi} observations of extragalactic radio sources \citep{luzum2010}.

The models and the software routines were be obtained from the \gls{acr:iers} via the \gls{acr:sofa} service website\footnote{\url{http://www.iausofa.org/}, accessed on January 8, 2020.} and implemented into the numerical 
propagation in \neptune. A description of the basic methods and algorithms is presented in the following.

%------------------------------------------------------------------------------
\subsection{Earth Orientation Parameters (EOP)}
\label{sec:eop}
%------------------------------------------------------------------------------

The reduction of celestial coordinates, which is described in \sect{sec:propagation-frames-gcrf-itrf}, requires so-called \gls{acr:eop}, which are measured 
quantities and regularly updated by the \gls{acr:iers}. As forecasts for each of the included quantity is difficult, the input data table for \neptune has to be updated regularly. 
The following \gls{acr:eop} are used for the reduction:
\begin{description}
 \item[\gls{sym:pomx}] Displacement, in arcseconds, of the \gls{acr:irp} from the \gls{acr:cip} with the direction being along the \gls{acr:irm}, the meridian at \ang{0;;} with an offset of \SI{100}{\metre} to the east of the Greenwich meridian.
 \item[\gls{sym:pomy}] Displacement, in arcseconds, of the \gls{acr:irp} from the \gls{acr:cip} with the direction being along the meridian \ang{90;;} W. 
 \item[$\Delta\acrshort{acr:ut1}$] The difference between \gls{acr:ut1} and \gls{acr:utc}, which is kept within a range of \SI{1}{\second}:
  \begin{equation}
   \acrshort{acr:utc} = \acrshort{acr:ut1} - \Delta\acrshort{acr:ut1}.
  \end{equation}
 \item[\gls{sym:pndx}] Celestial pole offset added as a correction to \gls{sym:pnx}, including \gls{acr:fcn}.
 \item[\gls{sym:pndy}] Celestial pole offset added as a correction to \gls{sym:pny}, including \gls{acr:fcn}.
\end{description}
The \gls{acr:cip} is the axis of the Earth rotation and, by definition, normal to the true equator.

%------------------------------------------------------------------------------
\subsection{Conversion between GCRF and ITRF}
\label{sec:propagation-frames-gcrf-itrf}
%------------------------------------------------------------------------------

The transformation between the \gls{acr:gcrf} and the \gls{acr:itrf} is accomplished via three consecutive matrix multiplications, which might also be combined in a single 
rotation matrix:
\begin{equation}
 \kot{gcrf}{itrf} = \gls{sym:bpn-matrix}\left(\gls{sym:t}\right) \gls{sym:era-matrix}\left(\gls{sym:t}\right) \gls{sym:pom-matrix}\left(\gls{sym:t}\right), 
\end{equation}
here and in the following the notation of the transformation matrix means that the conversion takes place from the subscripted to the superscripted system, with an example for 
a radius vector conversion being:
\begin{equation}
 \gls{sym:radvec}_{\gls{acr:gcrf}} = \kot{gcrf}{itrf} \gls{sym:radvec}_{\gls{acr:itrf}}.
\end{equation}
Noting that the rotation matrices are orthogonal, the inverse, which is required for the converting back, from \gls{acr:gcrf} to \gls{acr:itrf}, is simply the transpose, so that 
it is sufficient to compute the matrices once for both operations:
\begin{equation}
 \gls{sym:radvec}_{\gls{acr:itrf}} = \kot{itrf}{gcrf} \gls{sym:radvec}_{\gls{acr:gcrf}} = \gls{sym:pom-matrix}\left(\gls{sym:t}\right)^T \gls{sym:era-matrix}\left(\gls{sym:t}\right)^T \gls{sym:bpn-matrix}\left(\gls{sym:t}\right)^T \gls{sym:radvec}_{\gls{acr:gcrf}}. 
\end{equation}

The time parameter \gls{sym:t} used for the coordinate transformations, according to \citet{luzum2010}, is defined as the number of Julian centuries (in \gls{acr:tt}) since 
the date 2000 January 1.5, while $\gls{sym:t}_{\acrshort{acr:tt}}$ is the current Julian day in \gls{acr:tt}:
\begin{equation}
 \gls{sym:t} = \frac{\gls{sym:t}_{\acrshort{acr:tt}} - 2451545.0\;\acrshort{acr:tt}}{\num{36525}}
\end{equation}
Terrestrial time (\acrshort{acr:tt}), sometimes also referred to as \gls{acr:tdt}, is the ``independent argument for apparent geocentric ephemerides'' \citep{seidelmann2006}. 
It uses the SI second and has a constant offset to \gls{acr:tai}:
\begin{equation}
 \acrshort{acr:tt} = \acrshort{acr:tai} + \num{32.184}^s
\end{equation}

The individual steps for the transformation, as well as the intermediate frames called \gls{acr:cirf} and \gls{acr:tirf}, are shown in \fig{fig:gcrf-itrf-conversion} 
and shall be looked at in detail in the following.

\begin{figure}[h!]
 \centering
 \def\lenleft{6mm}
\def\lenup{8mm}

\begin{tikzpicture}[%
    bluebox-mod/.style={%
	bluebox, 
	drop shadow,
	text width=60mm
	},
    greenbox-mod/.style={%
	greenbox, 
	drop shadow,
	minimum width=20mm,
	minimum height=12mm, 
	text width=60mm
	},
    node distance=\lenup and \lenleft,
    >= latex, transform shape, scale=0.8
]
  
   \coordinate (a1) at (0,0);
   
   \node[text=dark-green, right=of a1] (gcrf) {Geocentric Celestial Reference Frame (GCRF)};
   \coordinate (c1) at ($ (gcrf.west) + (16cm,0) $);
   \coordinate (ctemp) at ($ (gcrf.south) + (2cm,0mm) $);
   
   \node[greenbox-mod, below=of ctemp] (series) {Compute series for X, Y, s};
   \node[greenbox-mod, right=of series] (table) {Get X, Y, s from lookup table};
   
   \coordinate (c2) at ($ 0.5*(series.south) + 0.5*(table.south) $);
   
   \node[greenbox-mod, below=of c2] (bpn) {Precession-nutation matrix $\mathbf{Q}\left(t\right)$};
   
   \node[below=of bpn] (c3) {};
   \node[] at (a1 |- c3) (a2) {};
   \node[text=dark-green, right=of a2] (cirf) {Celestial Intermediate Reference Frame (CIRF)};
   \coordinate (c4) at ($ (cirf.west) + (16cm,0) $);
   
   \node[bluebox-mod, below=of c3] (era) {Earth rotation angle matrix $\mathbf{R}\left(t\right)$};
   
   \node[below=of era] (c5) {};
   \node[] at (a1 |- c5) (a3) {};
   \node[text=dark-blue, right=of a3] (tirf) {Terrestrial Intermediate Reference Frame (TIRF)};
   \coordinate (c6) at ($ (tirf.west) + (16cm,0) $);
   
   \node[bluebox-mod, below=of c5] (pom) {Polar motion matrix $\mathbf{W}\left(t\right)$};
   
   \node[below=of pom] (c7) {};
   \node[] at (a1 |- c7) (a4) {};
   \node[text=dark-blue, right=of a4] (itrf) {International Terrestrial Reference Frame (ITRF)};
   \coordinate (c8) at ($ (itrf.west) + (16cm,0) $);

   \node[] at (gcrf.south -| series.north) (b2) {};
   \draw[->] (b2) -- (series);

   \node[] at (gcrf.south -| table.north) (b3) {};
   \draw[->] (b3) -- (table);
   
   \draw[->] (series.south) |- (bpn.west);
   \draw[->] (table.south) |- (bpn.east);
   \node[] at (cirf.north -| bpn.south) (b1) {};
   \draw[->] (bpn) -- (b1);
   
   \node[] at (cirf.south -| era.north) (b4) {};
   \draw[->] (b4) -- (era);
   
   \node[] at (tirf.north -| era.south) (b5) {};
   \draw[->] (era) -- (b5);
   
   \node[] at (tirf.south -| pom.north) (b6) {};
   \draw[->] (b6) -- (pom);
   
   \node[] at (itrf.north -| pom.south) (b7) {};
   \draw[->] (pom) -- (b7);
 
  
  %------------------------------------------------------------
  %
  %  Backgrounds
  %
  %
  
  \begin{pgfonlayer}{background}
   \node[fill=light-green, rectangle, rounded corners=1mm, drop shadow, fit=(gcrf) (c1), transform shape=false ] {};
   \node[fill=very-light-blue, rectangle, rounded corners=1mm, drop shadow, fit=(tirf) (c6), transform shape=false] {};
   \node[fill=very-light-green, rectangle, rounded corners=1mm, drop shadow, fit=(cirf) (c4), transform shape=false] {};
   \node[fill=light-blue, rectangle, rounded corners=1mm, drop shadow, fit=(itrf) (c8), transform shape=false] {};
  \end{pgfonlayer}
  
\end{tikzpicture}

 \vspace{0.5cm}
 \caption{Scheme for the reduction of the celestial coordinates. The transformation between the GCRF and ITRF involving three matrices 
          and two intermediate frames. Also shown is the choice in methods between computing X, Y and s via their associated series or look them up in a 
          pre-processed table.\label{fig:gcrf-itrf-conversion}}
\end{figure}

From a computational point of view, the series representation of the precession-nutation coordinates \gls{sym:pnx}, \gls{sym:pny} and 
\gls{sym:scio} is quite demanding, as they have to evaluated at every force model call during the numerical integration. A very promising approach, which was 
also presented by \citet{coppola2009}, is to use tabulated values for those quantities and thus save a lot of computation time. This approach was 
also implemented and is presented in \sect{sec:pn-tabulation}.

\subsubsection{Precession and nutation (IAU 2006/2000)}
\label{sec:precession-nutation}

The matrix $\gls{sym:bpn-matrix}\left(\gls{sym:t}\right)$ converts a state from the \gls{acr:icrf} to the \gls{acr:gcrf}:
\begin{equation}
 \gls{sym:radvec}_{\gls{acr:gcrf}} = \gls{sym:bpn-matrix}\left(\gls{sym:t}\right) \gls{sym:radvec}_{\gls{acr:icrf}}.
\end{equation}
It accounts for the motion of the \gls{acr:cip} in the \gls{acr:gcrf} due to precession and nutation, which can be expressed as \citep{luzum2010}:
\begin{equation}
\gls{sym:bpn-matrix}\left(\gls{sym:t}\right) = \left( \begin{array}{ccc} 1-a\gls{sym:pnx}^2 & -a\gls{sym:pnx}\gls{sym:pny} & \gls{sym:pnx} \\ -a\gls{sym:pnx}\gls{sym:pny} & 1-a\gls{sym:pny}^2 & \gls{sym:pny} \\ -\gls{sym:pnx} & -\gls{sym:pny} & 1-a(\gls{sym:pnx}^2+\gls{sym:pny}^2) 
\end{array} \right) \gls{sym:kot}_{3}\left(\gls{sym:scio}\right), \label{eq:PNtrans}
\end{equation}
with $\gls{sym:kot}_{3}\left(\gls{sym:scio}\right)$ being a rotation providing the position of the \gls{acr:cio} on the equator of the \gls{acr:cip} and 
the angle \gls{sym:scio} is therefore called the ``\gls{acr:cio} locator'' \citep{luzum2010}. The quantity $a$ is computed as \citep{luzum2010}:
\begin{equation}
 a = \frac{1}{1 + \sqrt{1-\gls{sym:pnx}^2-\gls{sym:pny}^2}} \cong \frac{1}{2} + \frac{1}{8}\left(\gls{sym:pnx}^2 + \gls{sym:pny}\right).
\end{equation}

The series for \gls{sym:pnx}, \gls{sym:pny} and \gls{sym:scio} are given in the following, each containing a polynomial and a trigonometric part. Note also, that the \gls{acr:cio} 
locator is a function of \gls{sym:pnx} and \gls{sym:pny}.

\begin{align}
\gls{sym:pnx} = &\ang{;;-0.016617} + \ang{;;2004.191898}\gls{sym:t} - \ang{;;0.4297829}\gls{sym:t}^2 - \ang{;;0.19861834}\gls{sym:t}^3 + \notag \\ 
                &+ \ang{;;0.000007578}\gls{sym:t}^4 + \ang{;;0.0000059285}\gls{sym:t}^5 + \notag \\
                &+ \sum_{i=1}^{1306}\left[A_{xs0,i} \sin\left(a_{p,i}\right) + A_{xc0,i} \cos\left(a_{p,i}\right)\right] + 
                   \sum_{i=1}^{253} \left[A_{xs1,i} \sin\left(a_{p,i}\right) + A_{xc1,i} \cos\left(a_{p,i}\right)\right]\gls{sym:t} + \label{eq:Xsum} \\
                &+ \sum_{i=1}^{36}  \left[A_{xs2,i} \sin\left(a_{p,i}\right) + A_{xc2,i} \cos\left(a_{p,i}\right)\right]\gls{sym:t}^2 +
                 + \sum_{i=1}^{4}   \left[A_{xs3,i} \sin\left(a_{p,i}\right) + A_{xc3,i} \cos\left(a_{p,i}\right)\right]\gls{sym:t}^3 +  \notag \\
                &+ \sum_{i=1}^{1}   \left[A_{xs4,i} \sin\left(a_{p,i}\right) + A_{xc4,i} \cos\left(a_{p,i}\right)\right]\gls{sym:t}^4  \notag \\
\notag \\
\gls{sym:pny} = & \ang{;;-0.006951} - \ang{;;0.025896}\gls{sym:t} - \ang{;;22.4072747}\gls{sym:t}^2 + \ang{;;0.00190059}\gls{sym:t}^3 + \notag \\
                    & + \ang{;;0.001112526}\gls{sym:t}^4 + \ang{;;0.0000001358}\gls{sym:t}^5 + \notag \\ 
                    & + \sum_{i=1}^{962}\left[A_{ys0,i} \sin\left(a_{p,i}\right) + A_{yc0,i} \cos\left(a_{p,i}\right)\right] + 
                        \sum_{i=1}^{277}\left[A_{ys1,i} \sin\left(a_{p,i}\right) + A_{yc1,i} \cos\left(a_{p,i}\right)\right]\gls{sym:t} + \label{eq:Ysum}  \\
                    & + \sum_{i=1}^{30} \left[A_{ys2,i} \sin\left(a_{p,i}\right) + A_{yc2,i} \cos\left(a_{p,i}\right)\right]\gls{sym:t}^2
                      + \sum_{i=1}^{5}  \left[A_{ys3,i} \sin\left(a_{p,i}\right) + A_{yc3,i} \cos\left(a_{p,i}\right)\right]\gls{sym:t}^3 + \notag \\
                    & + \sum_{i=1}^{1}  \left[A_{ys4,i} \sin\left(a_{p,i}\right) + A_{yc4,i} \cos\left(a_{p,i}\right)\right]\gls{sym:t}^4 \notag \\
\notag \\
\gls{sym:scio} = &-\frac{\gls{sym:pnx}\gls{sym:pny}}{2} + \ang{;;0.000094} + \ang{;;0.00380865}\gls{sym:t} - \ang{;;0.00012268}\gls{sym:t}^2 - \ang{;;0.07257411}\gls{sym:t}^3 +
\notag \\
                 & + \ang{;;0.00002798}\gls{sym:t}^4 + \ang{;;0.00001562}\gls{sym:t}^5 + \notag \\
                 & + \sum_{i=1}^{33}\left[A_{ss0,i} \sin\left(a_{p,i}\right) + A_{sc0,i} \cos\left(a_{p,i}\right)\right] +
                     \sum_{i=1}^{3} \left[A_{ss1,i} \sin\left(a_{p,i}\right) + A_{sc1,i} \cos\left(a_{p,i}\right)\right]\gls{sym:t} + \label{eq:ssum}\\
                 & + \sum_{i=1}^{25}\left[A_{ss2,i} \sin\left(a_{p,i}\right) + A_{sc2,i} \cos\left(a_{p,i}\right)\right]\gls{sym:t}^2 + 
                     \sum_{i=1}^{4} \left[A_{ss3,i} \sin\left(a_{p,i}\right) + A_{sc3,i} \cos\left(a_{p,i}\right)\right]\gls{sym:t}^3 + \notag \\
                 & + \sum_{i=1}^{1} \left[A_{ss4,i} \sin\left(a_{p,i}\right) + A_{sc4,i} \cos\left(a_{p,i}\right)\right]\gls{sym:t}^4 \notag
                    \end{align}

The arguments $a_{p,i}$ in the trigonometric series are a linear combination of \num{14} different terms to account for luni-solar and planetary nutation. 
The time variable in the following is also \gls{acr:tt}, while the original equations are based on \gls{acr:tdb}, the latter being defined as ``the independent 
argument of ephemerides and equations and motion that are referred to the barycenter of the solar system'' \citep{seidelmann2006}. Using \gls{acr:tt} instead of 
\gls{acr:tdb}, however, results in \gls{acr:cip} location errors less than \num{0.01} $\mu$as \citep{luzum2010}, which are negligible. The full equation, as provided by \citet{vallado2013}, using the 
Delaunay variables for the Sun and the Moon:
\begin{align}
 a_{p,i} &= a_{0x1,i} \gls{sym:manoMoon} + 
            a_{0x2,i} \gls{sym:manoSun} + 
            a_{0x3,i} \gls{sym:meanLongitudeMoon} +
            a_{0x4,i} \gls{sym:meanElongationMoon} + 
            a_{0x5,i} \gls{sym:meanRaanMoon} + \notag \\
           &+a_{0x6,i} \gls{sym:meanLong}_{\gls{idx:mercury}} +
            a_{0x7,i} \gls{sym:meanLong}_{\gls{idx:venus}} +
            a_{0x8,i} \gls{sym:meanLong}_{\gls{idx:earth}} +
            a_{0x9,i} \gls{sym:meanLong}_{\gls{idx:mars}} +
            a_{0x10,i} \gls{sym:meanLong}_{\gls{idx:jupiter}} +
            a_{0x11,i} \gls{sym:meanLong}_{\gls{idx:saturn}} + \label{eq:fundArg} \\
            &+a_{0x12,i} \gls{sym:meanLong}_{\gls{idx:uranus}} +
            a_{0x13,i} \gls{sym:meanLong}_{\gls{idx:neptune}} +
            a_{0x14,i} \gls{sym:generalPrecession} \notag
\end{align}
One has to be careful with the subscripts. The first subscript. here ``$0$'' (in general $0$ to $4$), corresponds to the summation terms in \eqs{eq:Xsum}{eq:ssum} matching those with the same index ($0$ to $4$). 
The ``$x$'' is for the coordinate \gls{sym:pnx}, while different coefficients are to be used for \gls{sym:pny} and \gls{sym:scio}. The third subscript denotes the fundamental argument under 
consideration (all \num{14} are listed above), while the last subscript ``$i$'' is for the summation, which is different, depending on which sum the argument 
belongs to in \eqs{eq:Xsum}{eq:ssum}.

The fundamental arguments in \eq{eq:fundArg} are subdivided into the luni-solar nutation terms, given with the Delaunay variables, and the planetary precession terms, with the 
following equations \citep{luzum2010}:
\begin{align}
 \gls{sym:manoMoon} &= \text{Mean anomaly of the Moon} \\
		    &= \ang{;;485868.249036} + \ang{;;1717915923.2178}\gls{sym:t} + \ang{;;31.8792}\gls{sym:t}^2 + \ang{;;0.051635}\gls{sym:t}^3 - \ang{;;0.00024470}\gls{sym:t}^4  \notag \\
		    \notag \\
 \gls{sym:manoSun}  &=\text{Mean anomaly of the Sun} \\
	      	    &= \ang{;;1287104.793048} + \ang{;;129596581.0481}\gls{sym:t} - \ang{;;0.5532}\gls{sym:t}^2 + \ang{;;0.000136}\gls{sym:t}^3 - \ang{;;0.00001149}\gls{sym:t}^4 \notag \\
		    \notag \\
 \gls{sym:meanLongitudeMoon} &= \text{Mean longitude of the Moon minus mean longitude of the ascending node of the Moon} \\
			     &= \ang{;;335779.526232} + \ang{;;1739527262.8478}\gls{sym:t} - \ang{;;12.7512}\gls{sym:t}^2 - \ang{;;0.001037}\gls{sym:t}^3 + \ang{;;0.00000417}\gls{sym:t}^4 \notag \\
		    \notag \\
  \gls{sym:meanElongationMoon} &= \text{Mean elongation of the Moon from the Sun} \\
			       &= \ang{;;1072260.703692} + \ang{;;1602961601.2090}\gls{sym:t} - \ang{;;6.3706}\gls{sym:t}^2 + \ang{;;0.006593}\gls{sym:t}^3 - \ang{;;0.00003169}\gls{sym:t}^4 \notag \\
		    \notag \\
  \gls{sym:meanRaanMoon} &= \text{Mean longitude of the ascending node of the Moon} \\
			 &= \ang{;;450160.398036} - \ang{;;6962890.5431}\gls{sym:t} + \ang{;;7.4722}\gls{sym:t}^2 + \ang{;;0.007702}\gls{sym:t}^3 - \ang{;;0.00005939}\gls{sym:t}^4 \notag \\
                    \notag \\
  \gls{sym:meanLong}_{\gls{idx:mercury}} &= \text{Mean longitude of Mercury / rad} = \num{4.402608842} + \num{2608.7903141574}\gls{sym:t}   \\
                    \notag  \\
  \gls{sym:meanLong}_{\gls{idx:venus}}   &= \text{Mean longitude of Venus / rad} = \num{3.176146697} + \num{1021.3285546211}\gls{sym:t}   \\
                    \notag  \\
  \gls{sym:meanLong}_{\gls{idx:earth}}   &= \text{Mean longitude of Earth / rad} = \num{1.753470314} + \num{628.3075849991}\gls{sym:t}   \\
                    \notag  \\
  \gls{sym:meanLong}_{\gls{idx:mars}}    &= \text{Mean longitude of Mars / rad} = \num{6.203480913} + \num{334.0612426700}\gls{sym:t}   \\
                    \notag  \\
  \gls{sym:meanLong}_{\gls{idx:jupiter}} &= \text{Mean longitude of Jupiter / rad} = \num{0.599546497} + \num{52.9690962641}\gls{sym:t}   \\
                    \notag  \\
  \gls{sym:meanLong}_{\gls{idx:saturn}} &= \text{Mean longitude of Saturn / rad} = \num{0.874016757} + \num{21.3299104960}\gls{sym:t}   \\
                    \notag  \\
  \gls{sym:meanLong}_{\gls{idx:uranus}} &= \text{Mean longitude of Uranus / rad} = \num{5.481293872} + \num{7.4781598567}\gls{sym:t}   \\
                    \notag  \\
  \gls{sym:meanLong}_{\gls{idx:neptune}} &= \text{Mean longitude of Neptune / rad} = \num{5.311886287} + \num{3.8133035638}\gls{sym:t}   \\
                    \notag  \\
  \gls{sym:generalPrecession} &= \text{General accumulated precession in longitude / rad} = \num{0.024381750}\gls{sym:t} + \num{0.00000538691}\gls{sym:t}^2
\end{align}

\subsubsection{Earth rotation angle}
\label{sec:earth-rotation-angle}

The Earth rotation angle matrix consists of a single rotation around the \gls{acr:cip}:
\begin{equation}
\gls{sym:era-matrix}\left(\gls{sym:t}\right) = \gls{sym:kot}_{3}(-\gls{sym:era}) 
\label{eq:ERAtrans}
\end{equation}
The \gls{acr:era} accounts for the sidereal rotation of the Earth, being the angle between \gls{acr:cio} and \gls{acr:tio} and defining \gls{acr:ut1} by convention \citep{luzum2010}. Thus, the time variable used to compute 
the \gls{acr:era} is the Julian day in \gls{acr:ut1} with an offset to the epoch 2000 January 1.5:
\begin{equation}
 \gls{sym:t} = \text{Julian Day \acrshort{acr:ut1}} - \num{2451545.0},
\end{equation}
so that \gls{acr:era} can be computed as:
\begin{equation}
 \gls{sym:era}\left(\gls{sym:t}\right) = 2\gls{sym:pi} \left(\num{0.7790572732640} + \num{1.00273781191135448}\gls{sym:t}\right).
\label{eq:ERA}
\end{equation}
From the \gls{acr:eop} the term $\Delta \gls{acr:ut1}$ is used to obtain \gls{acr:ut1} from \gls{acr:utc}, which is generally the input time system used. 

\subsubsection{Polar motion}
\label{sec:polar-motion}

The rotational matrix describing the polar motion, which is the difference between the \gls{acr:cip} and the \gls{acr:irp} is computed via three consecutive rotations:
\begin{equation}
\gls{sym:pom-matrix}\left(\gls{sym:t}\right) = \gls{sym:kot}_{3}\left(-\gls{sym:stio}\right) \gls{sym:kot}_{2}\left(\gls{sym:pomx}\right) \gls{sym:kot}_{1}\left(\gls{sym:pomy}\right), 
\label{eq:PM}
\end{equation}
here, \gls{sym:pomx} and \gls{sym:pomy} are the polar coordinates of the \gls{acr:cip} in the \gls{acr:itrf} (see also \sect{sec:eop}) and \gls{sym:stio} is the so-called \gls{acr:tio} locator, that ``provides the position of the TIO on the equator of the CIP corresponding to the kinematical definition of the 'non-rotating' origin (NRO) in the ITRS when the CIP 
is moving with respect to the ITRS due to polar motion.'' \citep{luzum2010}. While \gls{sym:pomx} and \gls{sym:pomy} are quantities subject to measurements and provide alongside with the other \gls{acr:eop}, the 
quantity \gls{sym:stio} results from a numerical integration of those values. However, it can be approximated using the average values for the \textit{Chandler wobble}, \gls{sym:chandlerWobble}, and the \textit{annual wobble}, \gls{sym:annualWobble}, of the pole \citep{vallado2013}, evaluated with respect 
to \gls{acr:tt}:
\begin{equation}
 \gls{sym:stio} = \ang{;;0.0015}\left(\frac{\gls{sym:chandlerWobble}^2}{\num{1.2}} + \gls{sym:annualWobble}^2\right)\gls{sym:t}_{\gls{acr:tt}} \cong -\ang{;;0.000047}\gls{sym:t}_{\gls{acr:tt}}
\end{equation}
%\begin{itemize}
% \item GCRS (see above) is geocentric but with same orientation as ICRS. GCRS in motion around the Earth-Moon barycenter. This is known as the Coriolis effect, which is included in PN model \cite{seidelmann2002}
 \todo[Texts for Chapter: Reference frames]{Write about ocean tides, atmospheric influences, crust tides and FCN. FCN causes an offset of 0.1 to 0.3 mas \citep{luzum2010}, ocean tides result in some cm offset in the positioning \citep{bradley2011}}
%\end{itemize}

%\begin{table}[h]
	%\caption{Table containing values of the amplitudes A and characterized multipliers N for each sum term, used for the determination of X, Y and $s_{CIO}$. The data\protect\footnotemark[1] is shown as in table $5.2a$, $5.2b$ and $5.2d$.}
	%\centerline{
	%\setlength{\tabcolsep}{1.5pt}
		%\begin{tabular}{ >{$}c<{$} >{$}c<{$} >{$}c<{$} >{$}c<{$} >{$}c<{$} >{$}c<{$} >{$}c<{$} >{$}c<{$} >{$}c<{$} >{$}c<{$} >{$}c<{$} >{$}c<{$} >{$}c<{$} >{$}c<{$} >{$}c<{$} >{$}c<{$} >{$}c<{$} }
			%\hline
			%i & A_{s0_{i}} & A_{c0_{i}} & N_{0x1_{i}} & N_{0x2_{i}} & N_{0x3_{i}} & N_{0x4_{i}} & N_{0x5_{i}} & N_{0x6_{i}} & N_{0x7_{i}} & N_{0x8_{i}} & N_{0x9_{i}} & N_{0x10_{i}} & N_{0x11_{i}} & N_{0x12_{i}} & N_{0x13_{i}} & N_{0x14_{i}} \\
			%%\hline
			 %1 & -17206424.18 & 3338.60 & 0 & 0 & 0 & 0 & 1 & 0 & 0 & 0 & 0 & 0 & 0 & 0 & 0 & 0 \\
			%\vdots & \vdots & \vdots & \vdots & \vdots & \vdots & \vdots & \vdots & \vdots & \vdots & \vdots & \vdots & \vdots & \vdots & \vdots & \vdots & \vdots \\
			 %1306 & 0.11 & 0.00 & 0 & 0 & 4 & -4 & 4 & 0 & 0 & 0 & 0 & 0 & 0 & 0 & 0 & 0\\
			%\hline
			%i & A_{s1_{i}} & A_{c1_{i}} & N_{1x1_{i}} & N_{1x2_{i}} & N_{1x3_{i}} & N_{1x4_{i}} & N_{1x5_{i}} & N_{1x6_{i}} & N_{1x7_{i}} & N_{1x8_{i}} & N_{1x9_{i}} & N_{1x10_{i}} & N_{1x11_{i}} & N_{1x12_{i}} & N_{1x13_{i}} & N_{1x14_{i}} \\
			 %1307 & -3309.73 & 205833.11 & 0 & 0 & 0 & 0 & 1 & 0 & 0 & 0 & 0 & 0 & 0 & 0 & 0 & 0 \\
			%\vdots & \vdots & \vdots & \vdots & \vdots & \vdots & \vdots & \vdots & \vdots & \vdots & \vdots & \vdots & \vdots & \vdots & \vdots & \vdots & \vdots \\
			%1559 & 0.00 & -0.10 & 1 & -1 & -2 & -2 & -1 & 0 & 0 & 0 & 0 & 0 & 0 & 0 & 0 & 0 \\
			%\hline
			%i & A_{s2_{i}} & A_{c2_{i}} & N_{2x1_{i}} & N_{2x2_{i}} & N_{2x3_{i}} & N_{2x4_{i}} & N_{2x5_{i}} & N_{2x6_{i}} & N_{2x7_{i}} & N_{2x8_{i}} & N_{2x9_{i}} & N_{2x10_{i}} & N_{2x11_{i}} & N_{2x12_{i}} & N_{2x13_{i}} & N_{2x14_{i}} \\
			%\vdots & \vdots & \vdots & \vdots & \vdots & \vdots & \vdots & \vdots & \vdots & \vdots & \vdots & \vdots & \vdots & \vdots & \vdots & \vdots & \vdots \\
			%\hline
		%\end{tabular}
	%}
	%\label{tab:TabulatedData}
%\end{table}
%test

%------------------------------------------------------------------------------
\subsubsection{Fast computation of precession-nutation coordinates using tabulated data}
\label{sec:pn-tabulation}
%------------------------------------------------------------------------------

The evaluation of the complete series in \eqs{eq:Xsum}{eq:ssum} is computationally expensive, especially in view of the possibility of pre-processing the values for 
\gls{sym:pnx}, \gls{sym:pny} and \gls{sym:scio} for a certain time period, as proposed by \citet{coppola2009}, for example. A tabulation of daily data would not result in 
spectral content being lost, as periodic motions with periods less than two days are conventionally excluded from precession-nutation \citep{coppola2009}.

A method to extract daily tabulated data for \gls{sym:pnx}, \gls{sym:pny} and \gls{sym:scio} was designed by \citet{lange2014} and implemented in \neptune. For intermediate 
values, a Lagrange interpolation is performed, which currently is of 5$^{th}$ order. According to \citet{coppola2009}, the maximum deviation with respect to the series 
representation is expected for \gls{sym:pny} (for an analysed interval between \num{1975} and \num{2050}) with \num{30} $\mu$as and \gls{sym:pnx} with \num{27} $\mu$as, respectively. 
Changing to a 9$^{th}$ order interpolation would reduce the interpolation error for the same time frame to about \num{1} $\mu$as for both, \gls{sym:pnx} and \gls{sym:pny}
\citep{coppola2009}.

The tables were generated for a period between January 1, 1975 and January 1, 2050 with \textit{Txys}, a tool written by \citet{lange2014}, and available within the \neptune tool
suite. If the propagation epoch happens to be outside the specified regime, the propagation will continue with computing the series instead of using the table.

In \tab{tab:comparison-nutation-precession}, timing parameters are listed for various orbits and force models.
\begin{table}[h!]
 \centering
 \caption{Mean propagation time for different orbits with \neptune, comparing full series computation of the IAU 2000/2006 CIO approach 
 versus interpolated results. The full force model included 36$\times$36 geopotential, atmosphere, luni-solar gravity, SRP, tides and albedo. Computations performed on an
Intel\textsuperscript{\textregistered} Core\textsuperscript{\texttrademark} i7 (2.67 GHz). \label{tab:comparison-nutation-precession}}
 \begin{tabular}{@{}p{4cm}SSSSS@{}}
  \toprule
  {Configuration}               & {Perigee alti. / \si{\kilo\metre}}     & {Ecc.}  & {Incl. / deg} & {Full series / \si{\second}} & {Interpolation / \si{\second}} \\
  \midrule
  LEO, full, 1 week                   & 400.0                                  & 0.0     & 54.0          & 10.1                       & 4.6  \\
  LEO, full, 1 month                  & 800.0                                  & 0.0     & 98.6          & 37.4                       & 18.0 \\
  GTO, full, 1 month                  & 400.0                                  & 0.723   & 28.0          &  7.0                       & 3.1 \\
  GEO, full, 1 year                   & 35786.0                                & 0.0     & 0.0           &  9.0                       & 4.1 \\
  \midrule
  LEO, 2-body, 1 week                   & 400.0                                  & 0.0     & 54.0          &  1.7                      &  0.1 \\
  LEO, 2-body, 1 month                  & 800.0                                  & 0.0     & 98.6          &  7.7                      &  0.3 \\
  GTO, 2-body, 1 month                  & 400.0                                  & 0.723   & 28.0          &  3.6                       & 0.2 \\
  GEO, 2-body, 1 year                   & 35786.0                                & 0.0     & 0.0           &  4.3                       & 0.2 \\
  \bottomrule  
 \end{tabular}
\end{table}
 For a full force model, using a 5$^{th}$ order Lagrange interpolation for the tabulated data results in half the time required for the full series computation. 
 For a 2-body propagation, the computation time even reduces by a factor of about \num{20} using tabulated and interpolated data.




